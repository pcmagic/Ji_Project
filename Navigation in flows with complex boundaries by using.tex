% Copyright 2004 by Till Tantau <tantau@users.sourceforge.net>.
%
% In principle, this file can be redistributed and/or modified under
% the terms of the GNU Public License, version 2.
%
% However, this file is supposed to be a template to be modified
% for your own needs. For this reason, if you use this file as a
% template and not specifically distribute it as part of a another
% package/program, I grant the extra permission to freely copy and
% modify this file as you see fit and even to delete this copyright
% notice. 

\documentclass{beamer}
\usepackage{mathtools}
\usepackage{comment}
\usepackage{CJKutf8}
\usepackage{bm}
\usepackage{todonotes}
\usepackage{amssymb}
\usepackage{booktabs}
\usepackage{amsmath}
\usepackage{graphicx}
\usepackage{dcolumn}
\usepackage{xcolor}
\usepackage{upgreek}
\newcommand{\lb}[1]{{\color{blue}#1}}
\newcommand{\obs}[1]{{\color{red}#1}}
\newcommand{\ve}[1]{\ensuremath{\mbox{\boldmath$#1$}}}
\newcommand\figref{Fig.~\ref}
\newcommand\sctref{section~\ref}
\newcommand{\norm}[1]{\left\lVert#1\right\rVert}
\renewcommand{\eqref}[1]{equation (\ref{#1})}

% There are many different themes available for Beamer. A comprehensive
% list with examples is given here:
% http://deic.uab.es/~iblanes/beamer_gallery/index_by_theme.html
% You can uncomment the themes below if you would like to use a different
% one:
%\usetheme{AnnArbor}
%\usetheme{Antibes}
%\usetheme{Bergen}
%\usetheme{Berkeley}
%\usetheme{Berlin}
%\usetheme{Boadilla}
%\usetheme{boxes}
%\usetheme{CambridgeUS}
%\usetheme{Copenhagen}
%\usetheme{Darmstadt}
%\usetheme{default}
%\usetheme{Frankfurt}
%\usetheme{Goettingen}
%\usetheme{Hannover}
%\usetheme{Ilmenau}
%\usetheme{JuanLesPins}
%\usetheme{Luebeck}
\usetheme{Madrid}
%\usetheme{Malmoe}
%\usetheme{Marburg}
%\usetheme{Montpellier}
%\usetheme{PaloAlto}
%\usetheme{Pittsburgh}
%\usetheme{Rochester}
%\usetheme{Singapore}
%\usetheme{Szeged}
%\usetheme{Warsaw}


\title{Navigation in flows with complex boundaries by using Reinforcement Learning}

% A subtitle is optional and this may be deleted
\subtitle{None}

\author{Zhang Ji\inst{1}}
% - Give the names in the same order as the appear in the paper.
% - Use the \inst{?} command only if the authors have different
%   affiliation.

\institute[CSRC] % (optional, but mostly needed)
{
  \inst{1}%
  Beijing Computational Science Research Center
}
% - Use the \inst command only if there are several affiliations.
% - Keep it simple, no one is interested in your street address.

\date{weekly report, 2018~2019}
% - Either use conference name or its abbreviation.
% - Not really informative to the audience, more for people (including
%   yourself) who are reading the slides online

\subject{Computational Biophysics Science}
% This is only inserted into the PDF information catalog. Can be left
% out. 

% Delete this, if you do not want the table of contents to pop up at
% the beginning of each subsection:
\AtBeginSubsection[]
{
    \begin{frame}{Outline}
        \tableofcontents[currentsection,currentsubsection]
    \end{frame}
}

% Let's get started
\begin{document}
\begin{frame}
	\titlepage
\end{frame}

\begin{frame}{Outline}
	\tableofcontents
	% You might wish to add the option [pausesections]
\end{frame}

\section{2018/12/14, 20}
\subsection{nothing}
\begin{frame}{The Jeffery equation}
  \begin{align}
		\dot{p}_n   & = \Omega_{nj}p_j + \lambda(S_{nj}p_j-p_np_kS_{kj}p_j) \\
		\Omega_{nj} & = \frac{1}{2}(u_{i,j}-u_{j,i})                        \\
		S_{nj}      & = \frac{1}{2}(u_{i,j}+u_{j,i})
  \end{align}

  \cite{jeffery1922motion,chevillard2013orientation,parsa2012rotation,einarsson2015angular}
\end{frame}

\begin{frame}{The Jeffery equation}
  \begin{equation}
		\dot{p}_n = \Omega_{nj}p_j + \lambda(S_{nj}p_j-p_np_kS_{kj}p_j), 
  \end{equation}
  meaning:\cite{einarsson2015angular}
  \begin{itemize}
    \item The first termmeans that the particle is rotated by the local flow vorticity, and that this rotation is independent of the particle shape.
    \item The second termmeans that the local rate-of-strain attracts the particle axis to the strongest eigendirection of $S_{nj}$.
    \item The strength of the attraction is affected by the particle shape. The more elongated an axis is, the stronger the attraction becomes. 
    \item $S_{nj}p_j$ is the stress on the $\bm{p}$ direction. 
    \item The non-linear part $p_np_kS_{kj}p_j$ s simply the
    stretching component of the strain. 
  \end{itemize}
\end{frame}

\begin{frame}{origion of equation}
  % \begin{equation}
  %   \begin{cases}
  %     \nabla\cdot\bm{u} &= 0, \\
  %     \mu\triangle\bm{u} &= \nabla p-\bm{f}\delta(\bm{x}_f), 
  %   \end{cases}
  % \end{equation}
  \begin{equation}
    \begin{cases}
      u_{i, j} &= 0, \\
      \mu u_{i,jj} &= p_{,i} - f_i \delta(\bm{x}_f), 
    \end{cases}
  \end{equation}
  \begin{equation}
    u_i(\bm{x}) = S_{nj}x_j + \Omega_{nj}x_j, 
  \end{equation}

  Assumptions: 
  \begin{itemize}
    \item apart from the disturbance produced in the immediate neighbourhood of the particle; 
    \item varies in space on a scale which is large compared with the dimenisions of the particle; 
  \end{itemize}
\end{frame}

\begin{frame}{set up of the problem}
  To discribe a microswimmer:
  \begin{itemize}
    \item head: major and minor radii of ellipse $rs_1$, $rs_2$;
    \item tail: major and minor radii of helix $rt_1$, $rt_2$, pitch $\lambda$ and $\#$ of pitch $n$;
    \item motion constant, one of: $\omega_t$, $\omega_m=\omega_h-\omega_t$, $F_t$, $T_m=T_h=-T_t$, $v_s$, (power: $p_m=\omega_m T_m$); 
    \item translational and rotational velocities at infinite space: $u_{inf}$ and $\omega_{inf}$; 
  \end{itemize}
  To discribe a shear flow:
  \begin{itemize}
    \item shear rate: $\uptau$; 
    \item shear norm: $\bm{t}_{shear}=\bm{z}$; 
    \item shear direction: $\bm{n}_{shear}=\bm{x}$; 
  \end{itemize}
\end{frame}

\begin{frame}{parameter selection}
  \begin{itemize}
    \item let: $rs_2 \equiv 1$, normalize $rs_1$, $rt_1$, $rt_2$, $\lambda$; 
    \item let: $\omega_m \equiv 1$, normalize $u_{inf}$, $\omega_{inf}$, $\uptau$; 
    \item let: $\alpha = \dfrac{rs_1}{rs_2}$, $\beta_1 = \dfrac{rs_1}{n \lambda}$, $\beta_2 = \dfrac{\uptau}{\omega_m}$
    \item Jeffery model only includes $\alpha$, 
    \item $\beta_1 \nearrow\ \leadsto$ ellipse; 
    \item $\beta_2 \nearrow\ \leadsto$ without spin; 
  \end{itemize}
\end{frame}

\begin{frame}
  \begin{figure}[htb]
    \centering
    \includegraphics[width=\columnwidth]{fig_20181220_traject.pdf}
  \end{figure}
\end{frame}

\begin{frame}
  \begin{figure}[htb]
    \centering
    \includegraphics[width=\columnwidth]{fig_20181220_traject2.pdf}
  \end{figure}
\end{frame}

\begin{frame}
  \begin{figure}[htb]
    \centering
    \includegraphics[width=\columnwidth]{fig_20181220_traject3.pdf}
  \end{figure}
\end{frame}

\begin{frame}{Try equation}
  \begin{equation}
    \dot{x}_i=\norm{v_s}p_i + \bm{u}^{b}_i
  \end{equation}
\end{frame}

\section{2019/02/11, passive and active ellipse}
\subsection{nothing}
\begin{frame}{The Jeffery + motion equation}
  \begin{align}
		\dot{p}_n &= \Omega_{nj}p_j + \lambda(S_{nj}p_j-p_np_kS_{kj}p_j) \\
		\Omega_{ij} &= \dfrac{1}{2}(u_{i,j}-u_{j,i})                        \\
    S_{ij} &= \dfrac{1}{2}(u_{i,j}+u_{j,i}) \\
    \lambda &= \dfrac{\alpha^2-1}{\alpha^2+1}, \alpha = \dfrac{rs_1}{rs_2} \\
    \dot{x}_i &= \norm{v_s}p_i + ub_i \\
    x_i(t) &= x_i(0) + \int_{0}^{t}\dot{x}_idt \\
    p_i(t) &= p_i(0) + \int_{0}^{t}\dot{p}_idt, 
  \end{align}
\end{frame}

\begin{frame}{The Stokes Equations}
  \begin{align}
    \mu u_{i,jj} &= p_{,i} - f_i \delta(\bm{x}_f) \\ 
    u_{i, j} &= 0 \\
    Some\ &boundary\ conditions...
  \end{align}
\end{frame}

\begin{frame}{Case 1: passive ellipse}
  \begin{figure}[htbp]
    \centering
    \includegraphics[page=1,width=0.8\textwidth,height=0.8\textheight,keepaspectratio]{fig_Navigation.pdf}
    \caption[caption]{Geometry of a ellipse. }
    \label{fig_problem1}
  \end{figure}
\end{frame}

\begin{frame}{Case 1: passive ellipse, results}
  \begin{figure}[htbp]
    \centering
    \includegraphics[page=2,width=0.8\textwidth,height=0.8\textheight,keepaspectratio]{fig_Navigation.pdf}
    \label{fig_problem2}
  \end{figure}
\end{frame}

\begin{frame}{Case 1: passive ellipse, results}
  \begin{figure}[htbp]
    \centering
    \includegraphics[page=3,width=0.8\textwidth,height=0.8\textheight,keepaspectratio]{fig_Navigation.pdf}
    \label{fig_problem3}
  \end{figure}
\end{frame}

\begin{frame}{Case 1: passive ellipse, results}
  \begin{figure}[htbp]
    \centering
    \includegraphics[page=4,width=0.8\textwidth,height=0.8\textheight,keepaspectratio]{fig_Navigation.pdf}
    \label{fig_problem4}
  \end{figure}
\end{frame}

\begin{frame}{Case 1: passive ellipse, results}
  \begin{figure}[htbp]
    \centering
    \includegraphics[page=5,width=0.8\textwidth,height=0.8\textheight,keepaspectratio]{fig_Navigation.pdf}
    \label{fig_problem5}
  \end{figure}
\end{frame}

\begin{frame}{Case 2: active ellipse}
  \begin{figure}[htbp]
    \centering
    \includegraphics[page=6,width=0.8\textwidth,height=0.8\textheight,keepaspectratio]{fig_Navigation.pdf}
    \caption[caption]{Geometry of a active ellipse. }
    \label{fig_problem6}
  \end{figure}
\end{frame}

\begin{frame}{Case 2: passive ellipse, results}
  \begin{figure}[htbp]
    \centering
    \includegraphics[page=7,width=0.8\textwidth,height=0.8\textheight,keepaspectratio]{fig_Navigation.pdf}
    \label{fig_problem7}
  \end{figure}
\end{frame}

\begin{frame}{Case 2: passive ellipse, results}
  \begin{figure}[htbp]
    \centering
    \includegraphics[page=8,width=0.8\textwidth,height=0.8\textheight,keepaspectratio]{fig_Navigation.pdf}
    \label{fig_problem8}
  \end{figure}
\end{frame}

\begin{frame}{Case 2: passive ellipse, results}
  \begin{figure}[htbp]
    \centering
    \includegraphics[page=9,width=0.8\textwidth,height=0.8\textheight,keepaspectratio]{fig_Navigation.pdf}
    \label{fig_problem}
  \end{figure}
\end{frame}

\begin{frame}{Case 2: passive ellipse, results}
  \begin{figure}[htbp]
    \centering
    \includegraphics[page=10,width=0.8\textwidth,height=0.8\textheight,keepaspectratio]{fig_Navigation.pdf}
    \label{fig_problem9}
  \end{figure}
\end{frame}

\section{2019/02/11, passive helix}
\subsection{nothing}
\begin{frame}{Case passive helix}
  \begin{figure}[htbp]
    \centering
    \includegraphics[page=12,width=0.8\textwidth,height=0.8\textheight,keepaspectratio]{fig_Navigation.pdf}
    \label{fig_problem10}
  \end{figure}
  \cite{fu2012bacterial,fu2009separation}
\end{frame}

\begin{frame}{Case passive helix}
  \begin{figure}[htbp]
    \centering
    \includegraphics[page=13,width=0.8\textwidth,height=0.8\textheight,keepaspectratio]{fig_Navigation.pdf}
    \label{fig_problem11}
  \end{figure}
  \cite{fu2012bacterial,fu2009separation}
\end{frame}

\begin{frame}{Case passive helix}
  \begin{figure}[htbp]
    \centering
    \includegraphics[page=14,width=0.8\textwidth,height=0.8\textheight,keepaspectratio]{fig_Navigation.pdf}
    \label{fig_problem12}
  \end{figure}
  \cite{fu2012bacterial,fu2009separation}
\end{frame}

\begin{frame}{Case passive helix}
  \begin{figure}[htbp]
    \centering
    \includegraphics[page=15,width=0.8\textwidth,height=0.8\textheight,keepaspectratio]{fig_Navigation.pdf}
    \label{fig_problem13}
  \end{figure}
\end{frame}

\begin{frame}{Case passive helix}
  \begin{figure}[htbp]
    \centering
    \includegraphics[page=16,width=0.8\textwidth,height=0.8\textheight,keepaspectratio]{fig_Navigation.pdf}
    \label{fig_problem14}
  \end{figure}
\end{frame}

\begin{frame}{Case passive helix}
  \begin{figure}[htbp]
    \centering
    \includegraphics[page=17,width=0.8\textwidth,height=0.8\textheight,keepaspectratio]{fig_Navigation.pdf}
    \label{fig_problem15}
  \end{figure}
\end{frame}

\begin{frame}{Case passive helix}
  \begin{figure}[htbp]
    \centering
    \includegraphics[page=18,width=0.8\textwidth,height=0.8\textheight,keepaspectratio]{fig_Navigation.pdf}
    \label{fig_problem16}
  \end{figure}
\end{frame}

\begin{frame}{Case passive helix}
  \begin{figure}[htbp]
    \centering
    \includegraphics[page=19,width=0.8\textwidth,height=0.8\textheight,keepaspectratio]{fig_Navigation.pdf}
    \label{fig_problem17}
  \end{figure}
\end{frame}

\section{2019/02/11, Passive and active swimmer}
\subsection{nothing}
\begin{frame}{Passive and active swimmer}
  \begin{figure}[htbp]
    \centering
    \includegraphics[page=33,width=0.8\textwidth,height=0.8\textheight,keepaspectratio]{fig_Navigation.pdf}
    \label{fig_problem18}
  \end{figure}
\end{frame}

\begin{frame}{Passive swimmer}
  \begin{figure}[htbp]
    \centering
    \includegraphics[page=34,width=0.8\textwidth,height=0.8\textheight,keepaspectratio]{fig_Navigation.pdf}
    \label{fig_problem19}
  \end{figure}
\end{frame}

\begin{frame}{active swimmer}
  \begin{figure}[htbp]
    \centering
    \includegraphics[page=35,width=0.8\textwidth,height=0.8\textheight,keepaspectratio]{fig_Navigation.pdf}
    \label{fig_problem20}
  \end{figure}
\end{frame}

\section{2019/02/11, Try to make table}
\subsection{nothing}
\begin{frame}{Try to make table}
  \begin{figure}[htbp]
    \centering
    \includegraphics[page=21,width=0.8\textwidth,height=0.8\textheight,keepaspectratio]{fig_Navigation.pdf}
    \label{fig_problem21}
  \end{figure}
\end{frame}

\begin{frame}{Try to make table}
  \begin{figure}[htbp]
    \centering
    \includegraphics[page=22,width=0.8\textwidth,height=0.8\textheight,keepaspectratio]{fig_Navigation.pdf}
    \label{fig_problem22}
  \end{figure}
\end{frame}

\begin{frame}{Try to make table}
  \begin{figure}[htbp]
    \centering
    \includegraphics[page=23,width=0.8\textwidth,height=0.8\textheight,keepaspectratio]{fig_Navigation.pdf}
    \label{fig_problem23}
  \end{figure}
\end{frame}

\begin{frame}{Try to make table}
  \begin{figure}[htbp]
    \centering
    \includegraphics[page=24,width=0.8\textwidth,height=0.8\textheight,keepaspectratio]{fig_Navigation.pdf}
    \label{fig_problem24}
  \end{figure}
\end{frame}

\begin{frame}{Try to make table}
  \begin{figure}[htbp]
    \centering
    \includegraphics[page=25,width=0.8\textwidth,height=0.8\textheight,keepaspectratio]{fig_Navigation.pdf}
    \label{fig_problem25}
  \end{figure}
\end{frame}

\begin{frame}{Try to make table}
  \begin{figure}[htbp]
    \centering
    \includegraphics[page=26,width=0.8\textwidth,height=0.8\textheight,keepaspectratio]{fig_Navigation.pdf}
    \label{fig_problem26}
  \end{figure}
\end{frame}

\begin{frame}{Try to make table}
  \begin{figure}[htbp]
    \centering
    \includegraphics[page=27,width=0.8\textwidth,height=0.8\textheight,keepaspectratio]{fig_Navigation.pdf}
    \label{fig_problem27}
  \end{figure}
\end{frame}

\begin{frame}{Try to make table}
  \begin{figure}[htbp]
    \centering
    \includegraphics[page=28,width=0.8\textwidth,height=0.8\textheight,keepaspectratio]{fig_Navigation.pdf}
    \label{fig_problem28}
  \end{figure}
\end{frame}

\begin{frame}{Try to make table}
  \begin{figure}[htbp]
    \centering
    \includegraphics[page=29,width=0.8\textwidth,height=0.8\textheight,keepaspectratio]{fig_Navigation.pdf}
    \label{fig_problem29}
  \end{figure}
\end{frame}

\begin{frame}{Try to make table}
  \begin{figure}[htbp]
    \centering
    \includegraphics[page=30,width=0.8\textwidth,height=0.8\textheight,keepaspectratio]{fig_Navigation.pdf}
    \label{fig_problem30}
  \end{figure}
\end{frame}

\begin{frame}{Try to make table}
  \begin{figure}[htbp]
    \centering
    \includegraphics[page=31,width=0.8\textwidth,height=0.8\textheight,keepaspectratio]{fig_Navigation.pdf}
    \label{fig_problem31}
  \end{figure}
\end{frame}


\bibliographystyle{apalike}
\bibliography{weekly_ref.bib}
\end{document}


